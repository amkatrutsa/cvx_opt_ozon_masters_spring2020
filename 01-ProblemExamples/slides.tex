\documentclass{beamer}
\mode<presentation>
{\usetheme{default}
 \usecolortheme{default}
 \usefonttheme{default}
 \setbeamertemplate{navigation symbols}{}
 \setbeamertemplate{caption}[numbered]} 
\usepackage[russian]{babel}
\usepackage[utf8]{inputenc}
\usepackage[T2A]{fontenc}
\usepackage{graphicx}
\usepackage{subcaption}
\usepackage{amsmath,amssymb}
\usepackage{url,hyperref}
\hypersetup{colorlinks,linkcolor=,urlcolor=blue}
\newtheorem{Th}{Теорема}
\newtheorem{ru_def}{Определение}
\newcommand{\tr}{\mathrm{trace}}
\usefonttheme[onlymath]{serif}
\newcommand{\Q}{{\color{red}{\textbf{Q}}}}

\newcounter{saveenumi}
\newcommand{\seti}{\setcounter{saveenumi}{\value{enumi}}}
\newcommand{\conti}{\setcounter{enumi}{\value{saveenumi}}}

\resetcounteronoverlays{saveenumi}

%%%%%%%%%%%%%%%%%%%%%%%%%%%%%%%%%%%%%%%%%%%%%%
% Formatting for title page
\addtobeamertemplate{navigation symbols}{}{    \usebeamerfont{footline}%
    \insertframenumber \ /  \inserttotalframenumber }
    
\newcommand{\bx}{\mathbf{x}}
\newcommand{\by}{\mathbf{y}}
\newcommand{\bc}{\mathbf{c}}
\newcommand{\ba}{\mathbf{a}}
\newcommand{\bA}{\mathbf{A}}
\newcommand{\bX}{\mathbf{X}}
\newcommand{\bU}{\mathbf{U}}
\newcommand{\bV}{\mathbf{V}}
\newcommand{\bSigma}{\boldsymbol{\Sigma}}
\newcommand{\bbR}{\mathbb{R}}
\newcommand{\T}{\top}
\newcommand{\diag}[1]{\mathrm{diag}(#1)}
\DeclareMathOperator*{\argmin}{\arg\min}
    
    
\title[Семинар 1]{Оптимизация I\\ 
Семинар 1: Постановки задач оптимизации}
\author{Александр Катруца}
\institute{
\vspace{-0.5cm}
\begin{figure}
\centering
\includegraphics[scale=0.3]{../pics/logo}
\end{figure}
%\vspace{-1cm}
}
\date{10 февраля 2020 г.}
%%%%%%%%%%%%%%%%%%%%%%%%%%%%%%%%%%%%%%%%%%%%%%


\begin{document}
\begin{frame}
  \titlepage
\end{frame}

\begin{frame}{Методология}
Основные этапы использования методов оптимизации при решении реальных задач:
\begin{enumerate}[<+->]
\item Определение целевой функции
\item Определение допустимого множества решений
\item Постановка и анализ оптимизационной задачи
\item Выбор наилучшего алгоритма для решения поставленной задачи
\item Реализация алгоритма и проверка его корректности
\end{enumerate}

\end{frame}

\begin{frame}{Безусловная задача оптимизации}

Дана функция $f: \bbR^n \to \bbR$

\begin{equation*}
\min_{\bx \in \bbR^n} f(\bx)
\end{equation*}

\begin{itemize}
\item Простейший тип задач
\item Но даже такие задачи не всегда легко решить
\item Начнём далее именно с анализа задач безусловной оптимизации
\end{itemize}
\end{frame}

\begin{frame}{Регрессия}
Дан набор пар $(\bx_i, y_i)$, необходимо восстановить функцию $f: \bbR^n \to \bbR$ так чтобы $y_i \approx f(\bx_i)$.
\pause
\begin{itemize}
\item Что надо дополнительно задать? (Вспомните курс по машинному обучению)
\pause
\item Примеры
\pause
\item Свойства
\end{itemize}
\pause
\Q: что делать, если значения $\bx_i$ даны неточно? 
\end{frame}

\begin{frame}{Оценка параметров распределений}
Даны некоторые векторы $\bx_1,\ldots, \bx_k$.
Надо найти из какого распределения они взяты.

\pause

\begin{center}
Основной метод: метод максимального правдоподобия
\end{center}

\pause

\begin{itemize}[<+->]
\item Как параметризовать распределение?
\item Что такое правдоподобие?
\item Пример для нормального распределения
\end{itemize}

\end{frame}

\begin{frame}{Условная задача оптимизации}

\begin{equation*}
\begin{split}
& \min f(\bx)\\
\text{s.t. } & \bx \in \mathcal{D}
\end{split}
\end{equation*}

\begin{itemize}
\item Дискретные задачи, например $\mathcal{D} = \{ 0, 1 \}^n$ или $\mathcal{D} = \{ -1, 1 \}^n$
\item Непрерывные задачи
\begin{equation*}
\mathcal{D} = \{ \bx \mid f_i(\bx) = 0, \; g_k(\bx) \leq 0 \}
\end{equation*}
\end{itemize}

\end{frame}

\begin{frame}{Регрессия по неполным данным}
Дан набор данных $(\bx_i, y_i)$, $y_i \in \mathbb{R}$.
Однако теперь известно, что только первые $m$ измерений дали точное значение $y_i$.
Про остальные известна лишь оценка снизу.
Необходимо найти модель, по которой можно восстановить имеющиеся данные и учесть данные оценки на отсутствующие.
\pause
\begin{itemize}[<+->]
\item Какие переменные надо ввести?
\item Какая целевая функция?
\item Какие ограничения?
\end{itemize}

\end{frame}

\begin{frame}{Toy metric learning}

Дан набор данных $(\bx_i, y_i)$, где $y_i$~--- класс объекта $\bx_i$
Надо получить псевдометрику такую, что объекты одного класса окажутся ближе, чем объекты разных классов.

\pause
\begin{itemize}[<+->]
\item Какая параметризация?
\item Какая целевая функция?
\item Какие ограничения?
\end{itemize}
\end{frame}

\begin{frame}{Задача про склад}
Есть список складов и список магазинов. 
Надо организовать доставку товаров из складов так, чтобы все запросы магазинов были удовлетворены.
Как это сделать?

\begin{itemize}[<+->]
\item Что нам дано?
\item Что надо найти?
\item Какая целевая функция?
\end{itemize}

\end{frame}

\begin{frame}{Определение расписания процессора}
\begin{enumerate}
\item Задан пул задач, необходимое время старта и окончания, а также объём работы необходимой для решения каждой задачи
\item Даны ограничения на скорость процессора и границы её изменения между соседними моментами времени
\item Надо составить расписание выполнения задач так, чтобы потребляемая мощность была минимальна
\end{enumerate}

\pause

\begin{itemize}[<+->]
\item Что дано?
\item Как задать расписание?
\item Как задать целевую функцию?
\item Какие ограничения?
\end{itemize}

\end{frame}

\begin{frame}{Лучевая терапия}
Распределить интенсивность облучения участка ткани так, чтобы максимизировать поражения заданной области и минимизировать поражение остальных клеток.

\pause

\begin{itemize}[<+->]
\item Какие переменные даны, а какие являются решением?
\item Какие есть ограничения?
\item Какая целевая функция?
\end{itemize}
\end{frame}

%\begin{frame}{Generative adversarial networks}
%Как сделать из случайной матрицы изображение похожее на реальное?
%
%\pause
%
%\begin{itemize}[<+->]
%\item Как отделить реальное изображение от сгенерированного?
%\item Как сгенерировать изображение из случайной матрицы?
%\item Какая связь между генерацией и дискриминацией? 
%\end{itemize}
%
%\end{frame}

\begin{frame}{Выводы}
\begin{itemize}
\item Типы задач оптимизации
\item Шаги для формализации задачи
\item Неоднозначность формализации
\end{itemize}
\end{frame}

\end{document}

5.13
7.13
7.22
